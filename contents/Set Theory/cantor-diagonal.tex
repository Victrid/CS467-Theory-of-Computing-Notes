\section{Cantor's Diagonal Argument}

\begin{definition}[$\mathbb{N}^+$]
\(\mathbb{N}^+ = \bigcup_{i=0}^{\infty} \mathbb{N}^i\).
\end{definition}

\begin{definition}[\(\{0,1\}^\mathbb{N}\)]
\(\{0,1\}^\mathbb{N}\) be the set contains all of \(\{0,1\}^*\).
\end{definition}

\begin{theorem}[Cantor's diagonal argument]\label{th:cantor-digonal}
There is no bijection between $\{0,1\}^\mathbb{N}$ and $\mathbb{N}$.
\end{theorem}

\begin{proof}

Assume the bijection is performed as:

\begin{align*}
f(1) &= x_1^1 x_2^1 \dots x_n^1 \dots \\
f(2) &= x_1^2 x_2^2 \dots x_n^2 \dots \\
f(3) &= x_1^3 x_2^3 \dots x_n^3 \dots \\
\dots & \dots \\
f(n) &= x_1^m x_2^m \dots x_n^m \dots \\
\end{align*}

Select the diagonal $d = x_1^1x_2^2x_3^3\dots$ and negate bit-wise: \(\overline{d} = \overline{x_1^1}\overline{x_2^2}\overline{x_3^3}\dots\)

\(\forall k, f(k)\) cannot match \(\overline{d}\) , as the \(k\)-th bit does not
match.
\end{proof}

\begin{corollary}[\(\{0,1\}^\mathbb{N} \not \cong \mathbb{N}\)]
 \(\{0,1\}^\mathbb{N} \ge \mathbb{N}\)
\end{corollary}

\begin{proof}
\(\mathbb{N} \rightarrow \{0,1\}^\mathbb{N}\) by binary representation,
add a 1 at front to recognize padding zeroes.

\(\{0,1\}^\mathbb{N} \rightarrow \mathbb{N}\) cannot be found according to theorem \ref{th:cantor-digonal}.
\end{proof}

\subsection{Relationship between \texorpdfstring{\(\mathbb{R}\) and
\(\B^\mathbb{N}\)}{R, and B\^N}}
\begin{corollary}
[\(\mathbb{R} \cong \{0,1\}^\mathbb{N}\)]
\end{corollary}

\begin{proof}
for $\mathbb{R} \rightarrow \{0,1\}^{\mathbb{N}}$:

We first find a bijection function that
\(\mathbb{R} \mapsto (0,1)\): \(f(x) = \frac{e^x}{e^x+1}\).

\((0,1) \rightarrow \{0,1\}^\mathbb{N}\): An injective function,
represent as float numbers in binary format. The 1x part is omitted as
all of them start with 0.

This function misses the trailing \(0\) (and \(1\)), so it's not
subjective. (It treats \(0.1000000\dots\) and \(0.0111111\dots\) as the
same)

For \(\{0,1\}^\mathbb{N} \rightarrow \mathbb{R}\): 

Represent as
\(0.x_1x_2x_3\dots\) but not as a binary float. Then \(0.01111\dots\)
and \(0.100000\dots\) are not the same.

Then with theorem \ref{th:shroeder}, \(\mathbb{R} \cong \{0,1\}^\mathbb{N}\).
\end{proof}

\section{Schr\"{o}der Bernstein
Theorem}
\begin{proof}

Let \(f: A\rightarrow B\), \(g: B \rightarrow A\) both injective.

\(f, g\) defines a directed graph: vertexes: \(A\cup B\), directed edge
\(f \cup g\)

This graph has every vertex \(d_{out} = 1\) and \(d_{in} \le 1\).

There are 4 types of connected components:

\begin{itemize}
\item
  cycles
\item
  Z component (infinite, with no start and end)
\item
  A starters
\item
  B starters
\end{itemize}

\begin{figure}[ht]
\centering
\begin{subfigure}[c]{0.3\textwidth}
         \centering
         \def\svgwidth{0.75\textwidth}
        \input{figures/shroder-circle.pdf_tex}
         \caption{Cycles}
     \end{subfigure}
     \begin{subfigure}[c]{0.3\textwidth}
         \centering
         \def\svgwidth{\textwidth}
        \input{figures/shroder-zcomponent.pdf_tex}
         \caption{Z component}
     \end{subfigure}

\begin{subfigure}[c]{0.3\textwidth}
         \centering
         \def\svgwidth{\textwidth}
        \input{figures/shroder-astarters.pdf_tex}
         \caption{A starters}
     \end{subfigure}
     \begin{subfigure}[c]{0.3\textwidth}
         \centering
         \def\svgwidth{\textwidth}
        \input{figures/shroder-bstarters.pdf_tex}
         \caption{B starters}
     \end{subfigure}
     
\caption{4 Connected Components}
\end{figure}

Squares are from \(A\), circles are from \(B\). solid edges are
from \(f\), dashed edges are from \(g\).

We can construct a bijective function \(A \mapsto B\): (Taken as bold edges):

\begin{itemize}
\item
  for cycles we take all \(f\) edges
\item
  for Z component we take all \(f\) edges
\item
  for A starters we take all \(f\) edges
\item
  for B starters we take all \(g\) edges reversely (\(g^{-1}\))
\end{itemize}

\end{proof}
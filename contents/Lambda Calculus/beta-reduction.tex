\section{Evaluating/Reducing \texorpdfstring{\(\lambda\)}{λ}-Terms: \texorpdfstring{\(\beta\)}{β}-Reduction}

\begin{definition}[$\beta$-reduction]
$\beta$-reduction is performed as 
$$(\lambda x.(f\;x)) y \rightarrow^{\beta} (f\;y)$$
\end{definition}

\begin{example}
A reduction:

\begin{align*}
&(\lambda f x.(x f))(\lambda g. g)(\lambda a.a)\\
\rightarrow^{\beta} &(\lambda x.(x (\lambda g.g)))(\lambda a.a)\\
\rightarrow^{\beta} &(\lambda a.a)(\lambda g.g) \\
\rightarrow^{\beta} &\lambda g.g
\end{align*}
\end{example}

\begin{definition}[redex]
\(A\rightarrow_{\beta} B\) then B is called redex.
\end{definition}

\begin{definition}[reduct]
\(A \rightarrow^*_{\beta} B\), if B cannot be reduced further, B is
called reduct. 

Also denoted as \(A \twoheadrightarrow^{\beta} B\).
\end{definition}



\(\beta\)-reduction does not necessarily terminate.

\begin{example}$\beta$-reducing \((\lambda x.(x\;x)) (\lambda x.(x\;x))\) will remain the same, while $\beta$-reducing \((\lambda x.(x\;x)) (\lambda y. ((y\;y)y))\) will grow larger every reduction.
\end{example}


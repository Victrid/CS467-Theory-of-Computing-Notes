\section{Natural Numbers}

We can use lambda calculus to denote natural numbers (in the Church way)

\begin{align*}
0 &= \lambda f x.x \\
1 &= \lambda f x.(f x) \\
n &= \lambda f x.f^{(n)} x
\end{align*}

\paragraph{Successor Function} For \(n = \lambda f x.f^{(n)} x\), we have
$$(n\;f) \rightarrow_{\beta} \lambda x. f^{(n)} x$$
$$(n\;f)(f\;x) \rightarrow_{\beta} \lambda f x.f^{(n+1)}x$$

Then we have the successor function:
$$succ = \lambda n. (\lambda f\;x.(n\;f)(f\;x))$$

\paragraph{Addition} For \(n = \lambda f x.f^{(n)} x\),
\(m = \lambda f x.f^{(m)} x\),
$$add = \lambda m\;n.\lambda f\;x .(n\;f\;(m\;f\;x))$$

Another way to do this: (as \(m\) are functions ready for taking a
function and a value)

$$addalt = \lambda m\;n.((m\; succ)\;n)$$

which means run \(succ\) for \(m\) times on \(n\).

\paragraph{Multiplication} Primitive recursive way:

$$mul=\lambda m\;n.((m\;(add\;n))\;\hat{0})$$

$$mulalt=\lambda m\;n\;f.(m\;(n\;f))$$


\paragraph{Exponentiation} Primitive recursive way:
$$\lambda m\;n.((n\;(mul\;m))\;\hat{1})$$

Or just $$(n\;m)$$ which is \(m^n\).

\section{\texorpdfstring{\(\lambda\)}{λ}-calculus}



\begin{definition}[\(\lambda\)-term]
We have an indefinite supply of variables. A \(\lambda\)-term is either:

\begin{itemize}
\item
  A variable \(f\)
\item
  An application \((A B)\), \(A,B\) are \(\lambda\)-terms, means
  \(A \leftarrow B\)
\item
  An abstraction \((\lambda x.A)\) where x is a variable, \(A\) is a
  \(\lambda\)-term.
\end{itemize}
\end{definition}


Question: Define a function that reads two input values and outputs the second one?

Here we have more than 1 input variable, but the abstraction can only
accept 1 input.

\((\lambda x.(\lambda y.y))\),taking the first variable to generate the
function that aims taking the second variable.

\textbf{Syntactic sugars}:

\begin{itemize}
\item
  \(\lambda x\;y.A\) instead of \(\lambda x.(\lambda y.A)\)
\item
  \(f:=(\lambda x\;y.A)\), write \((f\; s\; t)\) instead of
  \(((f\;s)t)\)
\item
  Drop the outermost parentheses. Write \(\lambda x.(f\;x)\) instead of
  \((\lambda x. (f\; x))\)
\end{itemize}

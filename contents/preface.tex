\section*{Preface}

This note is based upon the course \emph{CS467: Introduction to the Theory of Computing} given by Prof. D. A. Scheder at Shanghai Jiao Tong University during the 2021-2022 autumn semester. This class covers slightly different parts comparing to Prof. Long's class.

The questions are provided by Prof. Scheder as homework, and is done in groups. The solution you see is by group ``lambda'', which consists of Yehang Yu, Yazhou Tang, Weihao Jiang, and Yiming Dou, who withdrawed this class in the mid-term.

The course is divided into 5 parts, all trying to answer the question ``What is computation?'' They are:

\begin{itemize}
    \item Boolean Circuits
    \item Talking about Infinity
    \item Formalizing Computation
    \item Uncomputability
    \item Logic
\end{itemize}

In the first part \emph{Boolean Circuits}, we discussed about a simple computation model: Boolean circuits. Boolean computations are \emph{computing} something, but is finite. In other words, like an binary adder, we need to restrict how many bits are we calculating about.

In the second part \emph{Talking about Infinity}, we first discussed about formal languages, a way to describe an infinite set of strings with a finite set of rules. The definition of \emph{infinity} itself is then investigated.

In the third part \emph{Formalizing Computation}, we presented three computation models: primitive recursive functions, $\lambda$-calculus, and Turing machines. What problems they are able to compute and what are not are discussed.

In the forth part \emph{Uncomputability}, we focused onto Turing machines and problems it can't be used to solve. Several examples are discussed.

In the last part \emph{Logic}, We discussed about first order logic.
\section{Upper and Lower Bounds}

In this section, we discuss about the upper bounds of a fan-in 2
circuit's \(\land, \lor\) gates.

\subsection{Upper Bound}

\subsubsection{Truth Table Upper Bound}

\begin{theorem}
Truth table method gives a DNF formula with \(< 2^n\) items.
\end{theorem}

\begin{proof}
For \(\{x_0,\dots, x_n\}\):

Freeze the \(\{x_0,\dots, x_{n-1}\}\), check the last bit \(x_n\).

\begin{table}[ht]
    \centering
    \begin{tabular}{c|c}
    \hline
    Truth & Combination \\\hline
        0, 1 & one item in DNF \\\hline
0, 0 & no item in DNF \\\hline
1, 1 & 2 items can be bounded by removing \(x_{n}\) \\\hline
    \end{tabular}
\end{table}
\end{proof}

\begin{corollary}
Every \(f: \B^n \rightarrow \B\) has a DNF with
\(\le 2^{n-1}\) items.
\end{corollary}

We need one large \(\lor\)-gates with fan-in \(2^{n-1}\), and $2^{n-1}$ \(\land\)-gates with fan-in
  \(n\), i.e. \(2^{n-1}-1\) fan-in 2 \(\lor\)-gates and \((n-1)2^{n-1}\) fan-in 2 \(\land\)-gates.

The upper bound: \(n2^{n-1}-1\).

\subsubsection{Multiplexer Upper Bound}

The MUX function
$$f_n = ( f_{n-1} \land \lnot x_n )\lor (f'_{n-1} \land x_n)$$

which means if \(x_n\) then $f'_{n-1}$ else $f_{n-1}$.

Construct a tree recursively with MUX gate.

Depth $n-2$, and \(2^{n-1} -1\) MUX gates. MUX gate can be constructed as $(f_{true}\land \text{cond} )\lor (f_{false}\land \lnot \text{cond} )$, which needs 3 fan-in 2 gates.

The upper bound: \(3\times(2^{n-1} -1)\).

\subsubsection{Lupanov Upper Bound}

\begin{theorem}[Lupanov, 1958]
$$\text{\textsc{Size}}(C)_n \le \frac{2^n}{n}(1+o(1))$$
\end{theorem}

\begin{definition}[\(\mathbb{F}_2\) polynomial]
Parity of one or more clauses where clauses are conjunctions of literals. The clauses are called monomials.
\end{definition}

\begin{table}[ht]
        \centering
        \begin{tabular}{c|c}
        \hline
             Expressions & Construction \\\hline
             DNF & \(\lor\) of \(\land\)'s \\\hline
              CNF & \(\land\) of \(\lor\)'s \\\hline
              \(\mathbb{F}_2\) polynomial & \(\parity\) of \(\land\)'s \\\hline
        \end{tabular}
        \caption{Comparing Different Expressions}
        \label{tab:comp_f2}
    \end{table}

Use multiplication to imply $\land$, use modular addition to imply $\parity$.

\begin{example}
 A Boolean \(\mathbb{F}_2\) polynomial \(x + tz + xuv (\bmod 2)\).
\end{example}

For alphabet $\abs{\Sigma} = n$, we have \(2^n\) monomials and \(2^{2^n}\) polynomials.

\begin{theorem}
Every Boolean function \(f: \B^n \rightarrow \B\) can be written as a \(\mathbb{F}_2\) polynomial.
\end{theorem}

\begin{proof}
We can construct the transform table as \ref{tab:f2_transform}.

\begin{table}[ht]
    \centering
    \begin{tabular}{c|c}
    \hline
         Boolean function & \(\mathbb{F}_2\)-polynomial \\\hline
         \(x \land y\) & $xy$ \\\hline
  \(\lnot x\) & $1+x$ \\\hline
  \(x \lor y\) & $x+y+xy$ \\\hline
    \end{tabular}
    \caption{Transforming Boolean function to $\mathbb{F}_2$-polynomial}
    \label{tab:f2_transform}
\end{table}

The size of DNF clause can be shrinked as:
$$\Land \{\arr{x}\} = \lnot\Land\{\lnot x_i\} = 1 + \prod (1+x_i), \text{\textsc{Size}}(\land_n) \le 2n$$
\end{proof}

\begin{definition}
Let \(F \subseteq \B^n \rightarrow \B\). \(S(F)\) is the
  size of a smallest \(\parity-\land\) circuit $C$ that computes
  \(\forall f \in F\).
\end{definition}

\begin{lemma}\label{lm:polynomial-function-group}
let \(p \ge 1\) then
$$S(F) \le 2^n + \left\lceil \frac{2^n}{p} \right\rceil \cdot 2^p + \frac{\abs{F}\cdot 2^n}{p}$$
\end{lemma}

\begin{proof}
First construct all monomials.

A trivial summation could be large. Observe that if you already have a gate for \(x^I\) then to compute \(x^{I \cup \{i\}}\) only needs one more gate. Then we need less than \(2^n\) gates to compute all monomials.

Group monomials into $q$ groups, size $p$ each.

\begin{figure}[ht]
    \centering
    \def\svgwidth{0.75\textwidth}
    \input{figures/Lupanov.pdf_tex}
\end{figure}

We perform every addition in the monomial groups, which needs $2^p\;\parity$-gates for each group, which is $q\cdot 2^p$ in total. These polynomials are called in-group polynomials.

For each \(f \in F\), write \(f = \parity_{i \in K} T_i = \sum{Poly_i}\) where each $Poly_i$ is an in-group polynomial of group $i$. We need at most $q-1$ \(\parity\)-gates for each $f \in F$.

Then the total complexity should be $2^n$ for monomial calculation, $\left\lceil \frac{2^n}{p} \right\rceil \cdot 2^p$ for in-group polynomial, and $\abs{F} \cdot \left\lceil \frac{2^n}{p} \right\rceil$ for function computation.
\end{proof}

Now given $f(x_1,\dots,x_n)$ written as polynomial, select $m: 1\le m \le n$, factor out the variables $x_{m+1} \dots x_n$.

$$\sum{(factor)*monomials}$$

Where factor consists of $x_1,\dots,x_m$, and monomials consists of $x_{m+1},\dots,x_n$

Compute all the monomials by \(2^{n-m}\), and use the method mentioned in lemma \ref{lm:polynomial-function-group} to compute the factors. Finally compute the product-sum part by \(2*2^{n-m}\).

The total size:
$$\text{\textsc{Size}}(f) = 3\times 2^{n-m} + 2^m +\frac{2^m}{p}*2^p+\frac{2^{n}}{p}$$

Let $m = \sqrt{n}, p = n-m-\log n$, $\text{\textsc{Size}}(f) = \frac{2^n}{n}(1+o(1))$.

\subsection{Lower Bound}

\begin{theorem}[Shannon]
  There are functions
  \(f: \B^n \rightarrow \B\) with
  \(\text{\textsc{Size}}(f) \ge \cfrac{2^n}{2+2n}\).
\end{theorem}
\begin{proof}

\begin{lemma}
A circuit of size $s$ can be encoded as a string of length
\(\le s(2 + 2 \log s )\). By graph representation.
\end{lemma}

\begin{remark}
That means there are \(\le 2^{s(1+\log s)}\) circuits of size
\(\le s\).
\end{remark}

$$S \ge \frac{2^n}{2 + 2\log s} \ge \frac{2^n}{2+2n}$$

If \(s < \cfrac{2^n}{2+2n}\) then \(\exists f: S(f) \ge s\).

\end{proof}

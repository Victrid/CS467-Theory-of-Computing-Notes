\section{Basics}

Boolean circuits are constructed via \emph{Boolean basic functions}.

$\B^n \rightarrow \B$: $\lnot$, $\land$, $\lor$, $\parity$, NAND, NOR, etc.

\paragraph{3-Majority function $\Maj(x,y,z)$ and its circuits}

\begin{itemize}
\item \(\Land(x\lor y, x\lor z, y\lor z)\)
\item \(\Lor(x\land y, x\land z, y\land z)\)
\end{itemize}

\begin{remark}
The Majority function for more than 3 input cannot be induced.
\end{remark}

Their Truth table are shown in table \ref{tab:truth-table-of-major-function}:

\begin{table}[ht]
    \centering
    \begin{tabular}{c|c}
    \hline
         x y z & Result \\\hline
        \(x+y+z \le 1\) & 0 \\\hline
        \(x+y+z > 1\) & 1 \\\hline
    \end{tabular}
    \caption{Truth Table of $\Maj_3$}
    \label{tab:truth-table-of-major-function}
\end{table}

This shows that different \emph{circuits} can represent same \emph{function}.

\begin{theorem}[Boolean Circuit $\rightarrow$ Function]\label{th:boolean-circuit-soundness}
Every Boolean function can be computed by a circuit.
\end{theorem}

\begin{proof}
Let the Boolean function set \(B: \B^n \rightarrow \B\), define \(f_{\arr{t}}(\arr{x}) \in B := [ \arr{x} = \arr{t} ]\), which checks a single situation.

We have 
$$f_{\text{eye}_n}(\arr{x}) = x_n \land \Land_{i \neq n} \neg {x_i}$$
which represents only $x_n$ is meet.

Then a CNF sub-word can be represented as $f_t(\arr{x}) = \Land_t f_{eye_i}(\arr{x})$. The CNF can be constructed via $\Lor_i f_i(\arr{x})$.
\end{proof}

\begin{remark}[Equivalent DNF]
Let $\hat{f}_{\arr{t}}(\arr{x}) := [\arr{x} \neq \arr{t}]$, following similar procedures, and the DNF can also be constructed.
\end{remark}

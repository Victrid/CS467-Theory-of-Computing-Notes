\section{Validity is Undecidable}

\begin{theorem}[Undecidable First Order Logic]
Validity of First Order Logic sentences is undecidable.

There is no Turing machine $M$ that always halts and accepts \(x\) if and only if \(x\) is a valid sentence in First Order Logic.
\end{theorem}

\begin{theorem}[G\"{o}del's Incompleteness]
Validity of first order logic sentences is semi-decidable.

There is a Turing machine \(M\) that \(M\) accepts \(x\) if and only if \(x\) is a
sentence in first order logic and is valid. (But might fail to halt if it is not valid)
\end{theorem}

\begin{proof}

As proved in Post Correspondence Problem: A Turing Machine and an input
string can be defined as a tile set, and the machine halts if and only if $T$ has a
solution.

We now prove that first order logic can determine whether PCP has a solution.

Let PCP defined over \(\Sigma = \{0,1\}\),
\(T = \{\tile{t_1}{b_1}, \tile{t_2}{b_2}\dots\}\), where
\(t_i, b_i \in \{0,1\}^*\).

\(T\) defines a binary relation \(P\) on \(\Sigma^*\) via \((x,y) \in P \Leftrightarrow\) Tiling \(s=(s_1,\dots,s_m) \subset T^*\)
whith \(\text{top}(s)=x\), \(\text{bottom}(s)=y\).

The first order logic to determine whether this PCP has a solution:

Interpret the tile
$$\tile{101}{10}$$
as \(F_t = P(f_1\;f_0\;f_1\;e, f_1\;f_0\;e)\)

The tile combination can be interpreted as function combination:
$$G_t = \forall xy, P(x,y) \rightarrow P(f_a\;x, f_b\;y)$$

A solution of tile set is equivalent to: \(H: \exists x, P(x,x)\)

Then formula
\(F: (\land_{t \in \text{Tiles}} (F_t \land G_t)) \rightarrow H\)
describes whether this PCP tile set has a solution.

\begin{observation}
\(F\) is satisfiable, no matter the PCP tile set has a solution.
\end{observation}

\begin{theorem}
$F$ is valid $\Leftrightarrow$ PCP has a solution.
\end{theorem}

\begin{proof}

To prove $LHS \rightarrow RHS$:

\begin{lemma}\label{lm:nosol}
if PCP has no solution then there exists a model under which
$F$ evaluates to false.
\end{lemma}

\begin{proof}
The structure $M$ interprets everything in the intended way but sets
\((\epsilon, \epsilon) \not \in P_M\). 
\end{proof}

If PCP has no solution, then there is a structure M with $M \not \models F$ (lemma \ref{lm:nosol}).

To proove $RHS \rightarrow LHS$:

\begin{lemma}
If PCP instance has a solution then $F$ is valid.
\end{lemma}

\begin{proof}

Let $M$ be a structure. If \(\land_t(F_t\land G_t)\) evaluates to false, then $F$ evaluates to true. (As $false \rightarrow anything = true$)

Else, suppose $M \models \land_t(F_t\land G_t)$. We need to show that $M \models H$.

Proposition: if \(s = (s_1,\dots,s_n), s_i \in T\), \(\text{top}(s)=a\),
\(\text{bottom}(s)=b\), then \(M \models P(t_a,t_b)\).

Proofing by induction. If \(n=1\) then s is a tile \(\tile{a}{b}\) and
\(F_{\tile{a}{b}}\) is \(P(t_a,t_b)\).

Step: \(s=s_1s'\), \(s = s2,\dots,sn\) with bottom \(s' = b'\), top \(s' =
a'\).

By induction, $M \models P(t_a',t_b')$.

In particular $$M \models (P(t_a',t_b') \rightarrow
P(f_{a1}(t_a'),f_{b1}(t_b')))$$ so $M \models P(t_a,t_b)$.

Then we have the proposition establishes by induction.

\end{proof}
\end{proof}

\end{proof}
\section{Introduction}

Recall Finite state Automata on definition \ref{def:dfsa}.

Do the modification:
\begin{itemize}
\item
  Allow the FSA to stay, put and to go left
\item
  Allow the FSA to write on the memory
\end{itemize}

then it becomes the Turing Machine.

Turing Machine is described by the following things:

\begin{definition}[Turing Machine] A Turing machine consists of 
\begin{itemize}
\item
  \(\Sigma\) the input alphabet
\item
  \(\Gamma\) a working alphabet with \(\Sigma \subseteq \Gamma\)
\item
  \(\square\): Blank symbols \(\in \Gamma \backslash \Sigma\)
\item
  transition functions
  \(\delta: Q\times \Gamma \rightarrow Q \times \Gamma \times \{-1,0,1\}\)
\item
  Start state \(q_{start} \in Q\)
\item
  Accept state \(q_{accept} \in Q\)
\end{itemize}
\end{definition}


A Turing machine simulator can be found \href{https://www.turingmachinesimulator.com}{here}\footnote{https://www.turingmachinesimulator.com}.

\emph{Hint}: A Turing machine should never write blank in the middle of the
tape.
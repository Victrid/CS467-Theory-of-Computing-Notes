\section{Non-deterministic Turing Machine}

\subsection{\texorpdfstring{$\lambda$}{λ}-calculus}

\begin{theorem}
There is a Turing Machine that takes a
\(\lambda\)-term A and outputs \(B\) such that
\(A \rightarrow \succ_\beta B\) (if any).
\end{theorem}

\begin{proof}

Turing Machines only have finite alphabet (but \(\lambda\)-calculus have
infinite variable set): use indicators \(x_0, x_1, x_{10}\) to reduce
the alphabet.

\(\lambda\)-calculus have multiple ways to do simplification in
\(\beta\)-reduction. Like those Y-combinators, some of the
simplification route will not stop.

\begin{definition}[NDTM transition set]
\(\delta \subseteq Q \times \Gamma \times Q \times \Gamma \times \{L,S,R\}\).
\((q,s,q',s',d) \in \delta\) means the Turing Machine is allowed to
perform this transition (Other transition can also be done).

Non-deterministic \(\delta\) defines a binary relation rather than
functions on \(\Gamma^*\times Q \times \Gamma^*\).
\end{definition}

Convention: if \(q \in \{q_{accept}, q_{reject}\}\) then
\(\forall q' (q,s,q',s',d) \not \in \delta\) (Termination)

\begin{definition}[NDTM Accepts]
Non-deterministic Turing Machine \emph{accepts} \(x\) if
\(q_{start} x \overset{\delta}{\rightarrow}^* yq_{accept}z\).

The \emph{reject} state should not be defined. Otherwise one \(x\) can
be accepted and rejected, by different paths.
\end{definition}


\begin{theorem}
let \(M\) be a non-deterministic Turing Machine, there
is a deterministic Turing Machine \(M'\) with \(L(M) = L(M')\).
\end{theorem}

\begin{proof}
The deterministic Turing machine can be designed to try all
possibilities.

Let the \(l = \max_{q,s}|\delta(q,s)|\).

Add a choice tape and a work tape.

For every Phase:

\begin{itemize}
\item
  delete the work tape
\item
  increase the choice tape by 1 (like in binary, as maximum fan-out is
  limited), and check the status.
\end{itemize}

This performs as a broad-first search. It prevents that like entering a
infinite loop and can never get out of a state.
\end{proof}

\end{proof}
\section{Non Primitive Recursive Functions}

\begin{definition}[Ackermann Function]
\begin{align*}
    & A(0,y) &&= y+1 \\
    & A(x+1,0) &&= A(x,1) \\
    & A(x+1,y+1) &&= A(x,A(x+1,y))
\end{align*}

\end{definition}

Ackermann function grows very fast.

\begin{theorem}
Ackermann function is not primitive recursive.
\end{theorem}

\begin{proof}

\begin{lemma}
$A_i(y) := A(i,y)$, $\forall i, A_i$ is primitive recursive.
\end{lemma}

\begin{proof}
We have that $A_0 =succ$ is primitive recursive.

\begin{align*}
A_{i+1}(y) = \left\{
    \begin{aligned}
        & A_i(1) && y=0 \\
        & A_i(A_{i+1}(y-1)) && y \ge 1
    \end{aligned}
\right.
\end{align*}

If $A_i$ primitive recursive, then $A_{i+1}$ is primitive recursive.
\end{proof}


Goal: $\forall$ primitive recursive $f,\exists
i$ s.t. $A_i$ grows faster than $f$.

\begin{definition}[dominate]
$h: \mathbb{N} \rightarrow \mathbb{N}, f:\mathbb{N}^k \rightarrow N$, we say $h$ eventually dominates $f$ if $\exists n \in \mathbb{N}$ s.t. $f(\arr{x}) < h(\max(\arr{x},n)) = h(\norm{\arr{x}n})$.

We notate that $\norm{\arr{x}} := \max |x_i|$, $\norm{\arr{x}n} := \max(\arr{x}, n)$.

We notate that $f \succ g$ if $f$ dominates $g$.
\end{definition}

Goal:$\forall$ primitive recursive $f,\exists
i$ s.t. 

$A_i$ dominates $f$, i.e. $\exists i,n, A_i(\norm{\arr{x}n}) > f(\arr{x})$ for all $\arr{x}$.

$f$ is a basic primitive recursive function. We have that $A_1 \succ f$.

For combination $f=g(h_1(x), ..., h_k(x))$, assume $A_g \succ g$, $A_i \succ h_i$.

We define $$A_I \succ g, h_1, \dots, h_k\text{, where }I = \max\{g, i\}$$.

Then $$f \prec A_i(\max(A_i(\norm{xn}),n)) \preceq A_i(A_i(\norm{xn}))$$

We also have $A_{i+1}(y) = A_i(A_{i+1}(y-1)) \le A_i(A_i(y))$, then $f \prec A_{I+1}$.

For primitive recursion $f = Rec(g,h)$. assume $A_I \succ g,h$. $f(0,x) = g(x) \le A_I(\norm{xn})$.

$$f(t+1,x) = h(f(t,x),t,x) < A_I(\max(n, f(t,x), t, x))$$


\begin{lemma}
If $f = Rec(g,h)$, $A_I \succ g,h$, then $A_I^{t+1}(\norm{xn}) > f(t,x)$.
\end{lemma}

\begin{proof}
Induction by $t$.

$$f(t+1,x) < A_i(\max(n, f(t,x), t, x)) \le A_i(\max(n, A_i^{t+1}(\norm{xn}),t,x))$$

$$A_i(\max(n, A_i^{t+1}(\norm{xn}),t,x)) = A_i(A_i^{t+1}(\norm{xn})) = A_i^{t+2}(\norm{xn})$$
\end{proof}

\begin{lemma}
$A_{i}^{t+2}(\norm{xn}) < A_{i+2}(\norm{xnt})$.
\end{lemma}

\begin{proof}
\begin{gather*}
 A_{i+2}(\norm{xnt}) = A_{i+1}(A_{i+2}(\norm{xnt})) > A_{i+1}(A_{2}(\norm{xnt}))\\
 A_{i+1}(A_{2}(\norm{xnt})) = A_{i+1}(2\norm{xnt} +3) \ge A_{i+1}(\norm{xn}+t+2) \\
 A_{i+1}(\norm{xn}+t+2) = A_{i}(A_{i+1}(\norm{xn}+t+1))\\
 = A_i(A_i(A_{i+1}(\norm{xn}+t)))\\
 \cdots = A^{t+2}(A_{i+1}(y)) \ge A_i^{t+2}(\norm{xn})
\end{gather*}
\end{proof}

Then $A_{i+2} \succ Rec(g,h)$.
\end{proof}


\section{Finite State Automata}

\begin{definition}[Deterministic Finite State Automaton]\label{def:dfsa}
A deterministic finite state automaton consists of:

\begin{itemize}
\item \(\Sigma\) input alphabet
\item \(V\) set of states
\item \(S\in V\) a start state
\item \(A \subset V\) a set of accepting states
\item Transition function \(\delta: V \times \Sigma \rightarrow V\)
\end{itemize}
\end{definition}


Multi step transition \(\delta\) generalize to a function
\(\overset{\sim}{\delta}: V \times \Sigma^* \rightarrow V\) with
\(\overset{\sim}{\delta}(X, \epsilon) = X, \overset{\sim}{\delta}(X, a\omega)=\overset{\sim}{\delta}(\delta(X,a), \omega)\)

\begin{definition}[DFSA Accept, Recognize]
A DFSA accepts \(\omega \in \Sigma^*\) if
\(\overset{\sim}{\delta}(S, \omega) \in A\), a DFSA recognizes the language
\(\{\omega \in \Sigma^* | \text{FSA accepts }\omega\}\)
\end{definition}

\begin{definition}[non-deterministic FSA]
A non-deterministic finite state automaton consists of:
\begin{itemize}
\item
  \(\Sigma\) input alphabet
\item
  \(V\) set of states

\item
  \(S\subset V\) a set of start states
\item
  \(A \subset V\) a set of accepting states
  \item
  \(\delta \subset V \times \Sigma \times V \rightarrow V\),
  transition function set representing ``can go'' (non-deterministic) states
\end{itemize}
\end{definition}

\begin{definition}[NSA accepts]
An NSA accepts \(\omega \in \Sigma^*\) if \(\exists A \in S, B \in A\),
s.t. \(A \overset{\omega}{\rightarrow} B\).
\end{definition}


\subsection{FSA and Regular Languages}

\begin{theorem}
The language recognized by a FSA is regular.
\end{theorem}

\begin{proof}
For \(A \overset{x}{\rightarrow} B\) (\(\delta(A,x)=B\)) in automaton,
there is a grammar \(A \rightarrow xB\).
\end{proof}

\begin{theorem}
If \(L \subset \Sigma^*\) is a regular language then
there is a FSA recognizing it.
\end{theorem}

\begin{proof}
First we try to directly convert regular language to FSA:
\begin{itemize}
    \item Variables $V \sim$ states.
    \item \(D \rightarrow \epsilon \sim\) accepting states.
    \item start symbol \(E\sim\) start state.
    \item \(A \rightarrow xB \sim\) \(A \overset{x}{\rightarrow} B\)
(\(\delta(A,x)=B\)).
\end{itemize}

However, this is not a DFA!
\begin{observation}
Rules like
\begin{verbatim}
    C -> cD
    C -> cC
\end{verbatim}
The constructed \(\delta(C,c)\) have multi values.
\end{observation}

Direct translation constructs a non-deterministic finite state automata.

\subsection{Constructing FSA from a NSA}

For a NSA
\((V, S \subset V, A \subset V, \delta \subset V \times \Sigma \rightarrow V)\):

Construct FSA:
\((V' = \mathcal{P}(V), S' = {S} \in \mathcal{P}(V), A' = \{\omega \in V \mid \omega \cap V' \ne \emptyset \}, \delta(\{A_1, \cdots, A_k\}, x) = \cup (\delta(A_i,x)))\)
\end{proof}

\begin{corollary}
If L is regular then so is
\(\bar{L} = \Sigma^* \backslash L\).
\end{corollary}

\begin{proof}
DFA \((V,S,A,\delta) = \lnot (V,S,V\backslash A, \delta)\)
\end{proof}

When NSA have \(s\) states, with our construction, FSA will have no more
than \(2^s\) states.

Example: \(L\subset \{0,1\}^*\), with the 4th last bit be 1.

Construct with last 4 bit input ``shift'' automata, need 16 states.

\begin{theorem}
A FSA recognizing \(L_k \subset \{0,1\}^*, x_{-k}=1\) needs at
least $2^k$ states.
\end{theorem}

\begin{proof}
suppose having less than $2^k$ states:


\(\exists x,y \in \{0,1\}^i, x \ne y, \delta(S,x)=\delta(S,y)\).

Construct a sentence: \(x\omega = u0\omega, y\omega = u'1\omega\), \(|\omega|=k-1\). 

If \(x,y\) cannot be distinguished by \(\delta(S,\cdot)\), \(x\omega, y\omega\) could not be determined.
\end{proof}

\subsection{Proof \texorpdfstring{$L$}{L} is not a regular language (state
overflow)}

\(L \in \{0,1\}^*\) has the same number of 1's and 0's. The BNF can be defined as:
\begin{verbatim}
    S -> 0 S 1 S | 1 S 0 S | e
\end{verbatim}

Proof $L$ is not a regular language.
\begin{proof}\label{prf:infinite-states}
Suppose L is regular having FSA of $s$ states, consider
\(\{1, 11,\cdots, 1^{s+1}\}\).

$$\exists 1 \le i < j, \delta(S, 1^i) = \delta(S, 1^j) \text{ (The Pigeon hole)}$$

\(\delta(S, 1^i0^i)=\delta(S, 1^j0^i)\), which contradicts as
\(1^j0^i \not \in L\).
\end{proof}

\subsection{Regular Language and Regular Expressions}

If \(L_1\), \(L_2\) are regular then so are

\begin{enumerate}
\def\labelenumi{\arabic{enumi}.}
\item
  \(L_1\cup L_2\)
\item
  \(\overline{L_1}=\Sigma^* \backslash L_1\)
\item
  \(L_1 \cap L_2\)

  Can be also proved by
  \(\overline{\overline{L_1} \cup \overline{L_2}}\) with 1 and 2.
\item
  \(L_1\circ L_2\) if \(A \rightarrow \epsilon \in S(L_1)\), replace
  with \(A \rightarrow \text{\texttt{\_\_start\_\_}}(L_2)\).
\item
  \(L_1^*\)
\end{enumerate}

\begin{definition}[Regular Expression]
Expression $R$ is regular expression, if:
\begin{itemize}
\item
  $R$ is a symbol \(c \in \Sigma\)
\item
  $R$ is the empty string \(\epsilon\)
\item
  $R$ is \(R'^*\), where \(R'\) is regular expression
\item
  $R$ is \(R_1\circ R_2\), where \(R_1,R_2\) is regular expression
\item
  $R$ is \(R_1 \cup R_2\), also written as \((R_1 | R_2)\)
\end{itemize}
\end{definition}

\begin{theorem}[Kleene]
Every regular language has an equivalent regular
expression.
\end{theorem}

\section{Unrestricted Grammar}

\(a^nb^nc^m\) can be represented with a context-free grammar.

\(A^nB^n \rightarrow a^nb^n\):

\begin{verbatim}
    BC -> bc
    Bb -> bB
    cC -> Cc
\end{verbatim}

Unrestricted grammar is powerful!

\subsection{Proof \texorpdfstring{$L$}{L} is not a regular language}
\(L = \{a^nb^n \mid n \ge 0\}\) is not regular.

\begin{proof}[Proof with infinite states]
Refer to proof \ref{prf:infinite-states}.

If it were,there would be transition function \(\delta\):

\(\delta(S,a)\), \(\delta(S, aa)\), \(\delta(S, aaa)\), ... which have
infinite states.

Regular languages and FSA only have a finite memory to process
strings.
\end{proof}

\begin{proof}[Proof with infinite equivalent classes]

\begin{definition}[Equivalent Class]
A FSA \(A\) defines an equivalence relation
  \(\equiv_A\) on \(\Sigma^*\) via
  \(x\equiv_A y \Leftrightarrow \delta_A(S,x) =  \delta_A(S,y)\).
\end{definition}

\begin{observation}
\(\forall x,y,\omega \in \Sigma^*\), if
  \(x \equiv_A y\) then \(x\omega \equiv_A y\omega\).
\end{observation}

Whenever you have an equivalence relation \(\equiv\) on \(\Omega\), consider partitioning \(\Omega\) into equivalence classes.

The index of \(\equiv\) is the number of non-empty equivalence classes.

\begin{observation}
\(\equiv_A\) has index at most \(|Q|\), where
  \(Q\) is the set of the states.
\end{observation}

The language \(L = \{a^nb^n \mid n \ge 0\}\) have infinite equivalence
  classes.

\begin{definition}[Equivalence]
A language \(L \subseteq \Sigma^*\) defines an
  equivalence relation \(x \equiv_L y\) via
  \(x \equiv_L y \Leftrightarrow \forall \omega \in \Sigma^*: (x\omega\in L \leftrightarrow y\omega \in L)\).
\end{definition}

Given L and x, define \(L_x = \{\omega | x\omega \in L\}\).
  \(x \equiv_L y\) if \(L_x = L_y\).
  
\begin{observation}
If FSA A recognizes L then
  \(x \equiv_A y \Rightarrow x\equiv_L y\), in particular, \(\norm{\equiv_L} \le \norm{\equiv_A} \le |Q|\).
\end{observation}

\begin{theorem}[Regular and Equivalent Index]
If \(\equiv_L\) has finite index, then L is regular.
\end{theorem}

\begin{proof}
Equivalent classes can be derived to states. \(C_1, C_2 \in \equiv_L\),
\(x\in C_1, x\sigma \in C_2, \sigma \in \Sigma\), then the state
\(C_1 \overset{\sigma}{\rightarrow} C_2\).
\end{proof}

\begin{corollary}[Infinite Equivalent Index]
 If \(\norm{\equiv_L}\) is infinite the L is not
  regular.
\end{corollary}

index \(\norm{\equiv_L}\) is infinite means an infinite list of words, no
  two of which are equivalent.

a, aa, aaa, aaaa, ... are not equivalent in \(\equiv_L\), as
given \(a^ib^i \in L\) and \(a^ib^j \not \in L\), so the language $L$ is not regular.

\end{proof}

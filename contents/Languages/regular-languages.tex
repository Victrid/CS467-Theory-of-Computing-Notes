\section{Formal Languages}

How can we represent a valid e-mail address?

\begin{verbatim}
    Email -> Name @ Domain
    Name -> SingleCharacter Name
    Name -> SingleCharacter
    SingleCharacter -> `a' - `z', `-', `.'
\end{verbatim}

Find the BNF of Domain:

\begin{verbatim}
    NormCharacter -> `a' - `z', `-'
    dot -> `.'
    Leaf -> NormCharacter
    Leaf -> NormCharacter Leaf
    Domain -> Leaf dot Leaf
    Domain -> Leaf dot Domain
\end{verbatim}

Restrict a valid e-mail address more formally:

\begin{itemize}
\item
  Name not start or end with a `.'
\item
  Domain is a list of `.'-free names separated by `.'
\end{itemize}

The context-free grammar:

\begin{verbatim}
    E -> U @ D
    U -> C | C C | C S C
    S -> C | “.” | SS
    D -> N.L
    L -> N | N.L
    N -> C | CN
    C -> “a” - “z”
\end{verbatim}

\begin{definition}[Symbols]
The symbols are defines as:
\begin{itemize}
    \item $\rightarrow$: Rules of grammar
    \item $\Rightarrow$: Derivation
    \item $\Rightarrow^*$: Final derivation
\end{itemize}
\end{definition}


\(E \Rightarrow^* \text{a..b@me.cn}\)

\begin{definition}[Context-free Grammars]
Context-free grammars consists of
\begin{itemize}
    \item A set \(\Sigma\) of terminal symbols. 
    
    Example from above: \texttt{\{`a',\ `b',\ `@',\ `\_',\ `.',\ ...\}}
    \item A set \(V\) of variable symbols.
    
    Example from above: \texttt{\{E,\ U,\ D\}}
    \item A set \(R\) of rules, a rule is of the form
\(X \rightarrow u, X \in V, u \in (\Sigma \cup V)^*\).

    \item A start symbol \(S \in E\).
    
    Example from above: \texttt{E}
\end{itemize}
\end{definition}

CFG defines a relation \(\Rightarrow\) on the set \((\Sigma \cup V)^*\),
if
\(u, w \in (\Sigma \cup V)^*, X \in V\), Exists a rule \(X \rightarrow v, uXw \Rightarrow uvw\).

CFG forms like a tree, but email validation could have a simpler
grammar:

E =
\texttt{{[}\^{}\textbackslash{}.{]}.*{[}\^{}\textbackslash{}.{]}@({[}\^{}\textbackslash{}.{]}*\textbackslash{}.)*{[}\^{}\textbackslash{}.{]}*}, which is a list.

\begin{definition}[Regular Grammar]
a context-free grammar \((\Sigma, V, R, S)\) where only
rules in the form of \(X \rightarrow uY\) exists.
\end{definition}

\paragraph{Regular grammar for odd number of 1's}

\begin{verbatim}
    S -> 0 S | 1 D
    D -> e | 0 D | 1 S
\end{verbatim}

\paragraph{Context free grammar for more 1's than 0's}

\begin{verbatim}
    E -> e | 0 E 1 E | 1 E 0 E 
    B -> 1 B | 1 E | 0 E 1 B
\end{verbatim}

\paragraph{Context free grammar for sting containing 3n zeros}

\begin{verbatim}
    D -> 0 A | 1 D | e
    A -> 0 B | 1 A
    B -> 0 D | 1 B
\end{verbatim}

\paragraph{Context free grammar for more 1's than 0's, or sting containing 3n zeros}

\begin{verbatim}
    S -> E
    S -> D
    (with each rules above)
\end{verbatim}

\begin{theorem}[Combination]\label{th:regular-combination}
if $A, B$ are both regular
  languages, $A \cup B$ is regular, too.
\end{theorem}

\begin{proof}
\(A\cup B = (\Sigma, V_A\cup V_B \cup {S}, R_A \cup R_B, S')\), assume\(V_A\) and \(V_B\) are disjoint, \(S \not \in V_A \cup V_B\).
\end{proof}

\paragraph{Context free grammar for more 1's than 0's, and sting containing 3n zeros}

  Take the example of Chinese Remainder Theorem. \#1 is $0(\bmod 2)$, \#0 is
 $0(\bmod 3)$

\begin{verbatim}
    Zero_odd = 1 Zero_even | 0 One_odd | e
    Zero_even = 1 Zero_odd | 0 One_even
    One_odd = 1 One_even | 0 Two_odd
    One_even = 1 One_odd | 0 Two_even
    Two_odd = 1 Two_even | 0 Zero_odd
    Two_even = 1 Two_odd | 0 Zero_even
\end{verbatim}

\begin{theorem}[Common]
If $A, B$ are regular, then \(A \cap B\)
  is regular, too.
\end{theorem}
\begin{proof}
\begin{definition}[Normal Form]
 A regular grammar is in normal form if all rules are
in normal form.

\(P \rightarrow xQ, P \rightarrow \epsilon\) are in normal form.

\(P \rightarrow x, P \rightarrow Q\) are not in normal form.
\end{definition}
Assume all rules are in normal form.

  \begin{itemize}
  \item
    For \(\forall x \in \Sigma\) and rule
    \(P_A \rightarrow xQ_A \in A\), \(P_B \rightarrow xQ_B \in B\),

    Add the rule \(\tile{P_A}{P_B} \rightarrow x \tile{Q_A}{Q_B}\),
    with start symbol \(\tile{S_A}{S_B}\). The symbol
    \(\tile{a}{b}\) represents that the combined state of \(a\) and
    \(b\).
  \item
    \(\tile{P_A}{P_B} \Rightarrow^* \omega \in \Sigma^*\) if and only
    if \(P_A \Rightarrow^*_A \omega\) and \(\Rightarrow^*_B \omega\).
  \item
    If \(X \rightarrow \epsilon \in A\) and
    \(Y \rightarrow \epsilon \in B\), then
    \(\tile{X}{Y} \rightarrow \epsilon\).
  \end{itemize}
\end{proof}

\begin{lemma}[Regular Grammar Normal Form]
We can always transform a regular grammar into normal
form.
\end{lemma}

\begin{proof} For all the rules which is not normal form:
\begin{itemize}
    \item For \(P \rightarrow Q\), combine all states without variation.
    
    Compute all triples $(P,u,X)$ with \(P \Rightarrow^* uX\),
    \(P, X \in V, u \in \Sigma\).
    
    Add \(Q \rightarrow uX\) as a rule.
    \item For \(P \rightarrow x\), expand the rule to
\(P \rightarrow xP', P' \rightarrow \epsilon\)
    \item For \(P \rightarrow x_1x_2\cdots x_kB\), expand the rule as
\(A_{k-1} \rightarrow x_kA_k, k\in \{1, \cdots k\}\), with
\(A_0 = P, A_k = B\)
\end{itemize}

\end{proof}
\section{Context Free Grammars*}

\subsection{Ambiguity}

$x+y*z$ can be computed as $(x+y)*z$ or $x+(y*z)$, if no computation order is defined.

A grammar is called ambiguous if there's a word $w$ that has two
different parse trees.

\begin{example}
Two grammars:

\begin{verbatim}
    S -> S+S | S*S | (S) | x | y | z
\end{verbatim}

\begin{verbatim}
    E -> T | T+E
    T -> F | F*T
    F -> x | y | z | ( E )
\end{verbatim}

They both generate the same language, but only the second one is
unambiguous.
\end{example}

\begin{example}
Language \(\{a^mb^mc^n\} \cup \{a^mb^nc^n\}\), every grammar for this is
ambiguous.
\end{example}


\subsection{Properties of Context Free Grammars}

If \(L_1\) and \(L_2\) are context-free languages,
then so are:

\begin{itemize}
\item
  \(L_1 \cup L_2\)
\item
  \(L_1 \circ L_2\)
\item
  \(L_1^*\)
\item
  But not necessarily \(\overline{L_1}\)
\item
  \(L^R\), reversed. Just reverse all the rules.
\end{itemize}

But not necessarily \(L_1 \cap L_2\).

\begin{observation}
\(L_1 = \{a^mb^mc^n\}, L_2 = \{a^mb^nc^n\}\),
  \(L_1\cap L_2 = \{a^nb^nc^n\}\), which is not context-free.
\end{observation}
  
\begin{proof}

  if the longest path in the parse tree is larger than the number of
  variable symbols := q, we have
  \(V \Rightarrow^* xVz; V\Rightarrow^* y\). (pigeon hole)

  Every sufficiently long word \(\omega \in L\) can be written
  \(\omega = axyzb\), s.t. \(ax^iyz^ib \in L\), for \(i \ge 0\).
  (Pumping theorem)

  Take a longest path in the tree:
  \(S \rightarrow V_1 \rightarrow V_2 \cdots \rightarrow V_n \in V\)

  Then count \(q+1\) steps back: \(V_{n-1}, \cdots V_{n-q-1}\)

  However, we may have that
  \(V \rightarrow XVY, X,Y \rightarrow \epsilon\).

  A context-free grammar is in Chomsky normal Form if every rule is of
  one of following forms:

  \begin{itemize}
  \item
    \(A \rightarrow BC\)
  \item
    \(B \rightarrow b\)
  \item
    \(S \rightarrow \epsilon\), which is the start symbol.
  \end{itemize}
\begin{theorem}[CFG $\mapsto$ CNF]
If \(G\) is CFG then \(\exists G'\) in CNF with
  \(L(G) = L(G')\).
\end{theorem}

\begin{proof}
  \begin{itemize}
  \item
    \(A \rightarrow BCxDE\) replace by \(A \rightarrow BCXDE\),
    \(X \rightarrow x\)
  \item
    \(A \rightarrow BCDE\) replace by \(A \rightarrow NE\),
    \(N \rightarrow BCD\) ...
  \item
    \(C \rightarrow \epsilon\):

    Define $X$ is \emph{nullable} if \(X \Rightarrow^* \epsilon\).

    Algorithm to find:

    \(X_t\) := Symbol nullable with a parse tree of height \textless{}
    t.

    \(X_1 := \{ X | X \rightarrow \epsilon \}\)

    \(X_{t+1} := X_t \cup X \rightarrow YZ, Y,Z \in X_t\).

    Compute until \(X_{t+1} = X_t\)
  \end{itemize}

  Rule of conversion:

  for \(A \rightarrow BC\):

  Add \(A \rightarrow B\) if \(C\) is nullable, \(A \rightarrow C\) if
  \(B\) is nullable.

  Don't add \(A \rightarrow \epsilon\) except \(A\) is start symbol
  which is nullable.

  Now we have unary rules \(A \rightarrow B\). Delete
  \(A \rightarrow B\) and add \(X \rightarrow BC\) when
  \(X \rightarrow AC\) is a rule.

\end{proof}

% TODO

If height of the tree is no less than \(q+1\), there is a path of
length \(q+1\) with \(q+2\) vertices, where the last vertex is a leaf
and the \(q+1\) are variable symbols.

For the \(V \rightarrow V \rightarrow ... \rightarrow l\), The number
of leaves is at most \(2^q\).

If L is context free, then exists \(n\) (\(n = 2^{q+1}\), \(q\) is the
number of var of CNF) such that every word of length
\(|\omega| \ge n\) can be written as \(\omega = axyzb\) with
\(|xz| \ge 1 , |xyz| \le n\), and \(ax^iyz^ib \in L, \forall i\).

To proof that \(a^nb^nc^n\) is not context free, if it were, let n be
the constant from pumping lemma: \(a^nb^nc^n = lxyzr\) so Either
\(xyz \in {a,b}^*\) or \(\in {b,c}^*\) cannot contain all \(a, b, c\).

Then \(lxyzr\) in L, \(lx^2yz^2r\) not in L, which contradicts.
\end{proof}
